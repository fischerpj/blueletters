% Options for packages loaded elsewhere
\PassOptionsToPackage{unicode}{hyperref}
\PassOptionsToPackage{hyphens}{url}
\PassOptionsToPackage{dvipsnames,svgnames,x11names}{xcolor}
%
\documentclass[
  12pt,
]{letter}

\usepackage{amsmath,amssymb}
\usepackage{iftex}
\ifPDFTeX
  \usepackage[T1]{fontenc}
  \usepackage[utf8]{inputenc}
  \usepackage{textcomp} % provide euro and other symbols
\else % if luatex or xetex
  \usepackage{unicode-math}
  \defaultfontfeatures{Scale=MatchLowercase}
  \defaultfontfeatures[\rmfamily]{Ligatures=TeX,Scale=1}
\fi
\usepackage{lmodern}
\ifPDFTeX\else  
    % xetex/luatex font selection
    \setmainfont[]{Arial}
\fi
% Use upquote if available, for straight quotes in verbatim environments
\IfFileExists{upquote.sty}{\usepackage{upquote}}{}
\IfFileExists{microtype.sty}{% use microtype if available
  \usepackage[]{microtype}
  \UseMicrotypeSet[protrusion]{basicmath} % disable protrusion for tt fonts
}{}
\makeatletter
\@ifundefined{KOMAClassName}{% if non-KOMA class
  \IfFileExists{parskip.sty}{%
    \usepackage{parskip}
  }{% else
    \setlength{\parindent}{0pt}
    \setlength{\parskip}{6pt plus 2pt minus 1pt}}
}{% if KOMA class
  \KOMAoptions{parskip=half}}
\makeatother
\usepackage{xcolor}
\usepackage[left=4.2cm,right=2cm,bottom=2.8cm]{geometry}
\setlength{\emergencystretch}{3em} % prevent overfull lines
\setcounter{secnumdepth}{-\maxdimen} % remove section numbering
% Make \paragraph and \subparagraph free-standing
\makeatletter
\ifx\paragraph\undefined\else
  \let\oldparagraph\paragraph
  \renewcommand{\paragraph}{
    \@ifstar
      \xxxParagraphStar
      \xxxParagraphNoStar
  }
  \newcommand{\xxxParagraphStar}[1]{\oldparagraph*{#1}\mbox{}}
  \newcommand{\xxxParagraphNoStar}[1]{\oldparagraph{#1}\mbox{}}
\fi
\ifx\subparagraph\undefined\else
  \let\oldsubparagraph\subparagraph
  \renewcommand{\subparagraph}{
    \@ifstar
      \xxxSubParagraphStar
      \xxxSubParagraphNoStar
  }
  \newcommand{\xxxSubParagraphStar}[1]{\oldsubparagraph*{#1}\mbox{}}
  \newcommand{\xxxSubParagraphNoStar}[1]{\oldsubparagraph{#1}\mbox{}}
\fi
\makeatother


\providecommand{\tightlist}{%
  \setlength{\itemsep}{0pt}\setlength{\parskip}{0pt}}\usepackage{longtable,booktabs,array}
\usepackage{calc} % for calculating minipage widths
% Correct order of tables after \paragraph or \subparagraph
\usepackage{etoolbox}
\makeatletter
\patchcmd\longtable{\par}{\if@noskipsec\mbox{}\fi\par}{}{}
\makeatother
% Allow footnotes in longtable head/foot
\IfFileExists{footnotehyper.sty}{\usepackage{footnotehyper}}{\usepackage{footnote}}
\makesavenoteenv{longtable}
\usepackage{graphicx}
\makeatletter
\newsavebox\pandoc@box
\newcommand*\pandocbounded[1]{% scales image to fit in text height/width
  \sbox\pandoc@box{#1}%
  \Gscale@div\@tempa{\textheight}{\dimexpr\ht\pandoc@box+\dp\pandoc@box\relax}%
  \Gscale@div\@tempb{\linewidth}{\wd\pandoc@box}%
  \ifdim\@tempb\p@<\@tempa\p@\let\@tempa\@tempb\fi% select the smaller of both
  \ifdim\@tempa\p@<\p@\scalebox{\@tempa}{\usebox\pandoc@box}%
  \else\usebox{\pandoc@box}%
  \fi%
}
% Set default figure placement to htbp
\def\fps@figure{htbp}
\makeatother

\usepackage{booktabs}
\usepackage{caption}
\usepackage{longtable}
\usepackage{colortbl}
\usepackage{array}
\usepackage{anyfontsize}
\usepackage{multirow}
\makeatletter
\@ifpackageloaded{caption}{}{\usepackage{caption}}
\AtBeginDocument{%
\ifdefined\contentsname
  \renewcommand*\contentsname{Table des matières}
\else
  \newcommand\contentsname{Table des matières}
\fi
\ifdefined\listfigurename
  \renewcommand*\listfigurename{Liste des Figures}
\else
  \newcommand\listfigurename{Liste des Figures}
\fi
\ifdefined\listtablename
  \renewcommand*\listtablename{Liste des Tables}
\else
  \newcommand\listtablename{Liste des Tables}
\fi
\ifdefined\figurename
  \renewcommand*\figurename{Figure}
\else
  \newcommand\figurename{Figure}
\fi
\ifdefined\tablename
  \renewcommand*\tablename{Table}
\else
  \newcommand\tablename{Table}
\fi
}
\@ifpackageloaded{float}{}{\usepackage{float}}
\floatstyle{ruled}
\@ifundefined{c@chapter}{\newfloat{codelisting}{h}{lop}}{\newfloat{codelisting}{h}{lop}[chapter]}
\floatname{codelisting}{Listing}
\newcommand*\listoflistings{\listof{codelisting}{Liste des Listings}}
\makeatother
\makeatletter
\makeatother
\makeatletter
\@ifpackageloaded{caption}{}{\usepackage{caption}}
\@ifpackageloaded{subcaption}{}{\usepackage{subcaption}}
\makeatother

\ifLuaTeX
\usepackage[bidi=basic]{babel}
\else
\usepackage[bidi=default]{babel}
\fi
\babelprovide[main,import]{french}
\ifPDFTeX
\else
\babelfont{rm}[]{Arial}
\fi
% get rid of language-specific shorthands (see #6817):
\let\LanguageShortHands\languageshorthands
\def\languageshorthands#1{}
\usepackage{bookmark}

\IfFileExists{xurl.sty}{\usepackage{xurl}}{} % add URL line breaks if available
\urlstyle{same} % disable monospaced font for URLs
\hypersetup{
  pdfauthor={Pierre J. FISCHER},
  pdflang={fr},
  pdfsubject={Succession FISCHER née DURRENBERGER Aline (rapport des
libéralités)},
  colorlinks=true,
  linkcolor={blue},
  filecolor={Maroon},
  citecolor={Blue},
  urlcolor={Blue},
  pdfcreator={LaTeX via pandoc}}


\author{Pierre J. FISCHER}
\date{26 janvier 2026}

\begin{document}
\signature{Pierre J. FISCHER}
\address{Pierre J. FISCHER\\fishrp@pm.me +33 76748 2009\\39 rue de la
Figairasse, Bat I\\34070 MONTPELLIER\\F-}
\begin{letter}{Lotz Notaires Associés\\14 rue de Saverne,
PFAFFENHOFEN\\BP 40010\\67350 Val de Moder\\ ~ \\Objet: Succession
FISCHER née DURRENBERGER Aline (rapport des libéralités)}
\opening{Cher Maître, Chère Madame}


Je prends certes l'initiative de cette demande, mais sa portée
collective concerne l'ensemble des héritiers, dont tous profiteront.

Je vous prie ainsi de bien vouloir rapporter à la succession les
libéralités déguisées et les bénéficiaires identifiées par a) le
rapprochement des gros chèques et des factures afférentes aux
\textbf{travaux} b) l'évaluation de la mise à disposition gratuite du
\textbf{logement} sur une période de 35 ans au profit de Michèle Fischer
ainsi que le bénéfice du \textbf{mobilier restant} c) l'évaluation d'une
\textbf{maison} construite après la donation-partage du \textbf{terrain
nu} sis à Mertzwiller, 9 rue d'Eschbach et alloué en 1986 à Marguerite
Besnier.

Les détails de l'analyse comptable, les méthodes d'évaluation, les
arguments qualifiant les libéralités déguisées et/ou indirectes se
trouvent dans le document technique joint (22 pages).

Reprenant les valeurs ainsi établies pour les donations déguisées
accordées depuis 1986, il reste 129 K€ d'avoir en banque auxquels
s'ajoutent le débours de 41 K€ pour Marguerite Besnier. et 46 K€ pour
Michèle Fischer. nécessaire pour assurer le versement de 74 K€~à Aline
Salzeman. et 142 K€ à Pierre Fischer.

Nous avons d'ailleurs réduit dans ce calcul, le loyer de 180 à 160 K€
(-11\% de 35 années honorera l'assistance fournie à notre mère) et
similairement la valeur de la maison de 180 à 160 K€ puisque la valeur
du terrain nu est déjà contenue dans la donation-partage d'origine.
Pierre Fischer rapporte quant à lui 2340 € de dons.

NB: il va de soi que la maison rue d'Eschbach ne sera pas rapportée si
Marguerite Besnier établit que l'origine des fonds n'est pas parentale.

Le tableau ci-dessous apporte la preuve indiscutable des libéralités
déguisées, presqu'exclusivement en gros travaux et de l'iniquité
flagrante dans le traitement des héritiers. Nul ne pourra balayer pour
cause d'immatérialité un flux cumulé s'élevant à 580 K€.

\begin{table}
\caption*{
{\fontsize{20}{25}\selectfont  Formule de partage amiable\fontsize{12}{15}\selectfont }
} 
\fontsize{12.0pt}{14.0pt}\selectfont
\begin{tabular*}{\linewidth}{@{\extracolsep{\fill}}l|lrrr}
\toprule
 & nature & deja\_recu & quotepart & a\_recevoir \\ 
\midrule\addlinespace[2.5pt]
\multicolumn{5}{l}{mich\`ele} \\[2.5pt] 
\midrule\addlinespace[2.5pt]
sous\_total & — & -191.50 & 145.21 & -46.29 \\ 
\midrule 
 & loyer & -160.0 & 48.4 & -111.6 \\ 
 & mobilier & -30.0 & 48.4 & 18.4 \\ 
 & travaux & -1.5 & 48.4 & 46.9 \\ 
\midrule\addlinespace[2.5pt]
\multicolumn{5}{l}{marguerite} \\[2.5pt] 
\midrule\addlinespace[2.5pt]
sous\_total & — & -187.00 & 145.21 & -41.79 \\ 
\midrule 
 & travaux & -27.0 & 72.6 & 45.6 \\ 
 & maison & -160.0 & 72.6 & -87.4 \\ 
\midrule\addlinespace[2.5pt]
\multicolumn{5}{l}{aline} \\[2.5pt] 
\midrule\addlinespace[2.5pt]
sous\_total & — & -71.00 & 145.21 & 74.21 \\ 
\midrule 
 & travaux & -71.0 & 145.2 & 74.2 \\ 
\midrule\addlinespace[2.5pt]
\multicolumn{5}{l}{pierre} \\[2.5pt] 
\midrule\addlinespace[2.5pt]
sous\_total & — & -2.34 & 145.21 & 142.87 \\ 
\midrule 
 & don & -2.3 & 145.2 & 142.9 \\ 
\midrule 
\midrule 
reunion & — & -451.84 & 580.84 & 129 \\ 
\bottomrule
\end{tabular*}
\end{table}

En conséquence, je vous saurai gré de bien vouloir faire expressément
part à mes soeurs de cette proposition et de récolter l'expression de
leur accord. Ce qui ouvrira la voie à la rédaction d'un nouveau projet
de partage amiable.

Dans l'éventualité d'un accord introuvable, vous voudrez bien m'adresser
un procès-verbal de carence, qui en complément de la preuve présente des
diligences entreprises me permettra de saisir le tribunal sur la base de
la réunion fictive ci-dessus.



\closing{Recevez mes plus sincères salutations,}
\vfill
\encl{document technique}
\cc{lotz@notaires.fr, elodie.jud.67052@notaires.fr}
\end{letter}
\end{document}
