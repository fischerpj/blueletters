% Options for packages loaded elsewhere
% Options for packages loaded elsewhere
\PassOptionsToPackage{unicode}{hyperref}
\PassOptionsToPackage{hyphens}{url}
\PassOptionsToPackage{dvipsnames,svgnames,x11names}{xcolor}
%
\documentclass[
  french,
]{letter}
\usepackage{xcolor}
\usepackage[margin=1in,bottom=1in,left=1in,right=1in,left= 4.5cm, right=
3cm, top= 2.5cm, bottom= 3cm]{geometry}
\usepackage{amsmath,amssymb}
\setcounter{secnumdepth}{-\maxdimen} % remove section numbering
\usepackage{iftex}
\ifPDFTeX
  \usepackage[T1]{fontenc}
  \usepackage[utf8]{inputenc}
  \usepackage{textcomp} % provide euro and other symbols
\else % if luatex or xetex
  \usepackage{unicode-math} % this also loads fontspec
  \defaultfontfeatures{Scale=MatchLowercase}
  \defaultfontfeatures[\rmfamily]{Ligatures=TeX,Scale=1}
\fi
\usepackage{lmodern}
\ifPDFTeX\else
  % xetex/luatex font selection
\fi
% Use upquote if available, for straight quotes in verbatim environments
\IfFileExists{upquote.sty}{\usepackage{upquote}}{}
\IfFileExists{microtype.sty}{% use microtype if available
  \usepackage[]{microtype}
  \UseMicrotypeSet[protrusion]{basicmath} % disable protrusion for tt fonts
}{}
\makeatletter
\@ifundefined{KOMAClassName}{% if non-KOMA class
  \IfFileExists{parskip.sty}{%
    \usepackage{parskip}
  }{% else
    \setlength{\parindent}{0pt}
    \setlength{\parskip}{6pt plus 2pt minus 1pt}}
}{% if KOMA class
  \KOMAoptions{parskip=half}}
\makeatother
% Make \paragraph and \subparagraph free-standing
\makeatletter
\ifx\paragraph\undefined\else
  \let\oldparagraph\paragraph
  \renewcommand{\paragraph}{
    \@ifstar
      \xxxParagraphStar
      \xxxParagraphNoStar
  }
  \newcommand{\xxxParagraphStar}[1]{\oldparagraph*{#1}\mbox{}}
  \newcommand{\xxxParagraphNoStar}[1]{\oldparagraph{#1}\mbox{}}
\fi
\ifx\subparagraph\undefined\else
  \let\oldsubparagraph\subparagraph
  \renewcommand{\subparagraph}{
    \@ifstar
      \xxxSubParagraphStar
      \xxxSubParagraphNoStar
  }
  \newcommand{\xxxSubParagraphStar}[1]{\oldsubparagraph*{#1}\mbox{}}
  \newcommand{\xxxSubParagraphNoStar}[1]{\oldsubparagraph{#1}\mbox{}}
\fi
\makeatother


\usepackage{longtable,booktabs,array}
\usepackage{calc} % for calculating minipage widths
% Correct order of tables after \paragraph or \subparagraph
\usepackage{etoolbox}
\makeatletter
\patchcmd\longtable{\par}{\if@noskipsec\mbox{}\fi\par}{}{}
\makeatother
% Allow footnotes in longtable head/foot
\IfFileExists{footnotehyper.sty}{\usepackage{footnotehyper}}{\usepackage{footnote}}
\makesavenoteenv{longtable}
\usepackage{graphicx}
\makeatletter
\newsavebox\pandoc@box
\newcommand*\pandocbounded[1]{% scales image to fit in text height/width
  \sbox\pandoc@box{#1}%
  \Gscale@div\@tempa{\textheight}{\dimexpr\ht\pandoc@box+\dp\pandoc@box\relax}%
  \Gscale@div\@tempb{\linewidth}{\wd\pandoc@box}%
  \ifdim\@tempb\p@<\@tempa\p@\let\@tempa\@tempb\fi% select the smaller of both
  \ifdim\@tempa\p@<\p@\scalebox{\@tempa}{\usebox\pandoc@box}%
  \else\usebox{\pandoc@box}%
  \fi%
}
% Set default figure placement to htbp
\def\fps@figure{htbp}
\makeatother



\ifLuaTeX
\usepackage[bidi=basic]{babel}
\else
\usepackage[bidi=default]{babel}
\fi
% get rid of language-specific shorthands (see #6817):
\let\LanguageShortHands\languageshorthands
\def\languageshorthands#1{}


\setlength{\emergencystretch}{3em} % prevent overfull lines

\providecommand{\tightlist}{%
  \setlength{\itemsep}{0pt}\setlength{\parskip}{0pt}}



 


\makeatletter
\@ifpackageloaded{caption}{}{\usepackage{caption}}
\AtBeginDocument{%
\ifdefined\contentsname
  \renewcommand*\contentsname{Table des matières}
\else
  \newcommand\contentsname{Table des matières}
\fi
\ifdefined\listfigurename
  \renewcommand*\listfigurename{Liste des Figures}
\else
  \newcommand\listfigurename{Liste des Figures}
\fi
\ifdefined\listtablename
  \renewcommand*\listtablename{Liste des Tables}
\else
  \newcommand\listtablename{Liste des Tables}
\fi
\ifdefined\figurename
  \renewcommand*\figurename{Figure}
\else
  \newcommand\figurename{Figure}
\fi
\ifdefined\tablename
  \renewcommand*\tablename{Table}
\else
  \newcommand\tablename{Table}
\fi
}
\@ifpackageloaded{float}{}{\usepackage{float}}
\floatstyle{ruled}
\@ifundefined{c@chapter}{\newfloat{codelisting}{h}{lop}}{\newfloat{codelisting}{h}{lop}[chapter]}
\floatname{codelisting}{Listing}
\newcommand*\listoflistings{\listof{codelisting}{Liste des Listings}}
\makeatother
\makeatletter
\makeatother
\makeatletter
\@ifpackageloaded{caption}{}{\usepackage{caption}}
\@ifpackageloaded{subcaption}{}{\usepackage{subcaption}}
\makeatother
\usepackage{bookmark}
\IfFileExists{xurl.sty}{\usepackage{xurl}}{} % add URL line breaks if available
\urlstyle{same}
\hypersetup{
  pdfauthor={Pierre J. FISCHER},
  pdflang={fr},
  pdfsubject={Demande d'exonération des prélevements sociaux sur pension
française},
  colorlinks=true,
  linkcolor={blue},
  filecolor={Maroon},
  citecolor={Blue},
  urlcolor={Blue},
  pdfcreator={LaTeX via pandoc}}


\author{Pierre J. FISCHER}
\date{12 février 2026}
\begin{document}
\signature{Pierre J. FISCHER}
\address{Pierre J. FISCHER 1 58 02 67 512 036 14\\fishrp@pm.me +33 76748
2009\\39 rue de la Figairasse, Bat I\\MONTPELLIER 34070\\F-}
\begin{letter}{Centre de Gestion des Retraites\\DGFIP de la
Haute-Vienne\\à l'attention de Mme Valérie Durieux\\87043 LIMOGES
Cedex\\ ~ \\Subject: Demande d'exonération des prélevements sociaux sur
pension française}
\opening{Madame,}


Je suis résident fiscal français mais assujetti au régime de la sécurité
sociale de l'Office Européen des Brevets, établi à Munich.

Par ailleurs, je suis bénéficiaire de la pension civile personnelle
Numéro 3111020048771V servie par l'État français au titre de mes
fonctions antérieures de fonctionnaire. Et je constate que les
prélèvements sociaux (CSG, CRDS, CASA) sont opérés à la source sur cette
pension depuis l'origine.

En l'espèce, je suis assujetti au régime de la sécurité sociale de mon
dernier employeur, l'Office Européen des Brevets à Munich qui mandate
Cigna Healthcare pour en servir les prestations sociales. Je ne suis
donc pas affilié à la Sécurité sociale française et mes frais de santé
dans le cadre de ce régime étranger sont intégralement et exclusivement
supportés par Cigna Healthcare.

Or, conformément aux dispositions du règlement (CE) n° 883/2004 du
Parlement européen et du Conseil du 29 avril 2004, relatif à la
coordination des systèmes de sécurité sociale, une personne ne peut être
assujettie qu'à un seul régime obligatoire de sécurité sociale au sein
de l'Union européenne.

En invoquant ce principe communautaire de l'unicité de la législation
sociale découlant du réglement CE 1408-71, la législation française
prévoit expressément que les personnes affiliées à un régime obligatoire
de sécurité sociale d'un autre État membre de l'Union européenne, de
l'Espace économique européen ou de la Suisse sont exonérées de CSG et de
CRDS sur leurs revenus de source française (cf.~article L136-1 du Code
de la sécurité sociale, ainsi que la loi de financement de la sécurité
sociale pour 2019).

En conséquence, je vous prie de bien vouloir : 1. Procéder à la
cessation des prélèvements sociaux sur ma pension française. 2.
M'accorder le remboursement des sommes indûment prélevées au titre des
années antérieures, conformément aux dispositions applicables en matière
de réclamation contentieuse.

Vous trouverez ci-joint les pièces justificatives suivantes :

\begin{itemize}
\item
  Attestation d'affiliation au régime obligatoire étranger (Cigna --
  Office Européen des Brevets, Munich).
\item
  Copie de mon dernier relevé de pension mentionnant les prélèvements
  sociaux.
\item
  Copie de mon avis d'imposition sur les revenus 2024.
\end{itemize}

Au demeurant, mon avis d'imposition reflète déjà l'exonération de la CSG
sur les Revenus de Capitaux Mobiliers, comme en témoigne aussi la
décision de remboursement de la CSG + CRDS sur un rachat d'assurance-vie
opéré en 2021.

Je reste à votre disposition pour toute information complémentaire



\closing{et vous prie d'agréer, Madame, l'expression de mes salutations
distinguées,}
\vfill
\encl{\_CX01\_this\_csg\_exoneration\_158026751203614.pdf\\\_CX02\_2025\_11\_BPENS\_novembre.pdf\\\_CX03\_2551\_004106
health insurance
confirmation.pdf\\\_CX03\_\_H\_2144\_EPO\_ASSURANCE\_attestation.pdf\\\_CX04\_CIGNA\_InsuranceCertificate.pdf\\\_CX05\_Avis\_d\_impot\_\_sur\_les\_revenus\_et\_prelev\_sociaux\_2024.pdf\\\_CX06\_4135
- DGFIP\_Bastien\_acceptation partielle.pdf}
\end{letter}
\end{document}
